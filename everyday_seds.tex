\label{everyday_seds}
\section{SED Analysis: an overview}

Spectral energy distributions (SEDs) provide a wealth of knowledge for astronomical sources. With increasing datasets over the years, astronomers have been able to utilize broadband SEDs more frequently for their research. As such, many SED analysis packages have been created to help astronomers model, fit and derive physical quantities from their SED data. These packages exist in different flavors: inversion (examples), template-based (examples), Bayesian statistics, and other methods; also common are home-grown fitting routines (Bongiorno's, others). 

Most distributed fitting packages are tailored for specific data sets or spectral ranges (PAHFIT, STARLIGHT, ), providing robust fitting methods and results. They require the data to be in a specific format with specific units in order for the tool to work properly. When fitting a broadband SED that spans over decades in the spectrum, the astronomer may have to gather data from different public archives and team members to add to their own dataset. More often than not, the datasets are presented in different file formats and will most likely be in different units. The user must provide their own methods to extract the necessary data from each file, convert the units, and output a file in the single format supported by the tool. While these exacting steps are worthwhile for the quality fitting results, the amount of effort put in just gathering pre-processed data and plugging it in to a fitting engine should be unnecessary. The astronomer may also want to inspect the SED, for example plotting it against different units, normalizing some of points or spectral segments, shifting the SED to another redshift, and other simple visualization tasks.

The virtual observatory...

Iris offers a standard means of building large broadband SEDs from different sources in various data formats, while providing robust fitting methods and interactive visualization capabilities.