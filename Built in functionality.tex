\section{Built-in functionality}
\label{builtin}

In this section, we go through a specific use-case of Iris to showcase its main functionalities. We build, inspect, and fit a SED of the flat radio spectrum quasar (FRSQ) blazar PKS 1127-14, and save our results to file. In Section \ref{sec:components}, we describe how these functions work in terms of the Iris stack.

\subsection{Building the SED}

As stated in Section \ref{sec:everyday_seds}, Iris can read data from a variety of sources in different formats. In Figure \ref{fig:load_data}, we have loaded PLANCK data in the form of an ASCII spectrum file (where there is a column for the spectral, flux, and flux errors) and a photometry catalog-style SAMP message of WISE data from TOPCAT. When Iris receives data in non-VO compliant formats, Iris opens file reader GUIs in which the user provides the mapping for the spectral, flux and flux uncertainties. The file importers provide helpful hints for the user when filling in the forms.

A typical Iris session begins with loading SED data. In Figure \ref{fig:load_data}, PLANCK data in the form of an ASCII spectrum file (where there are columns for the spectral, flux, and flux errors), is read into Iris through an Import Setup frame. The user simply selects the columns containing the spectral, flux and flux uncertainties, and supplies the units for each from a given list of common units. The user also has unrpocessed data of PKS 1127-14 from the four WISE bands. He/She reads the data into TOPCAT [REFERENCE] to apply a complex correction to the photometric points, and then beams the data over to Iris via SAMP \ref{fig:samp_topcat}. The user loads the data as a Photometry catalog, using the help box at the bottom of the importer frame as a guide. 

Wanting to analyze the entire SED of the blazer, the user queries the NED database for photometric data through the NED SED Service portal in Iris. The data is automatically added to the SED Builder and the display. The user then opens the ASDC Data Center, which provides more control over the data being added to the plot; he/she types the target name, chooses observation date ranges, selects optical/UV data from SWIFT and GALEX, and imports the data to the SED. The results are shown in Figure \ref{fig:load_data}.

\subsection{Inspecting the SED}

With all of the data uploaded, the user may switch the spectral and flux units; Iris takes care of all unit conversions once the data is read in. Before starting a fitting session, the user wants to remove from the SED data without flux uncertainty measurements. The user opens the Metadata Browser from the plotting display (SED Viewer), switches to the Data tab, and types a Python-expression into the Boolean filter to highlight the points whose error measuments are above 0 (i.e. points that have uncertainty measurements). With the desired data points hilgihted in the window, the user clicks "Create new SED;" this adds another SED to the SED Builder, which is managed separately from our original dataset.

At this point, the user chooses to shift the SED to restframe. PKS 1127-14 is at redshift z=1.18. The user opens up the Science tool by clicking the "Shift, Interpolate, Integrate" icon on the Iris desktop; under "Redshift," he/she types 1.18 into the Initial field, and leaves 0 in the Final box, finally creating a new SED in the restframe.

The user decides to save the SED they built to a file. SEDs are saved in VO-compliant FITS or VOTable formats which can be re-read into Iris or other programs (e.g. TOPCAT; IDL or Python interpreters). Users can save all of the metadata associated with the SEDs, or choose to save just the spectral, flux and flux uncertainties in a simplier format.